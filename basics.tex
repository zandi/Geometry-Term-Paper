\section{Basics}

The undergraduate math student would be well prepared for this paper with
prior experience in algebra, linear algebra and geometry.
However, there are still various concepts and terms which
may be unfamiliar.
These will be briefly discussed here before the main content, which will make heavy
use of them. Note that a central concept to algebraic geometry,
the \emph{variety}, is not covered here, nor elsewhere in this paper.
It seems to be abstract enough that such introductory materials
don't bother with it. Despite this, the following should provide 
enough background for the following sections. Let's begin with
what is surely the most widely used and foundational concept;
\emph{Projective Space}.

\begin{mydef}
Given a Field $k$, the $\emph{Projective Space of Degree n over k}$, denoted $\mathbb{P}^n_k$
is the quotient $(k^{n+1} \setminus \{\bold{0}\}) / \sim$, where $\sim$ is the equivalence relation 
defined as 
\[
(x_0, ..., x_n) \sim (y_0, ..., y_n)~if~\exists \lambda \in k~s.t.~\lambda \neq 0,
(x_0, ..., x_n) = (\lambda y_0, ..., \lambda y_n)
\]
Given this, points in projective space are often denoted $P = [ x_0 : ... : x_n]$,
with the colons and brackets indicating this is an equivalence class. So then,
$[ y_0: ... : y_n ] = P = [ x_0 : ... : x_n ]$ if $(y_0, ..., y_n) \sim (x_0, ..., x_n)$.
\end{mydef}
The definition is abstract to purposely highlight the fact that, 
given a field $k$ we can create a projective space.
Effectively, the projective space is the set of all lines in 
$k^{n+1}$ through the origin. This is also why the projective space
has one fewer dimension than its corresponding vector space.

So, the ``points" in projective space are lines in the corresponding
vector space. Another interesting feature of projective space is
that parallel lines intersect at a ``point at infinity". This can
be thought of as making partitions of lines in the projective space
pased on parallelism, as if we were defining an equivalence relation.
Now, each equivalence class is assigned a unique ``point at infinity"
that each line in the equivalence class intersects. Thus, if two lines
are parallel, they intersect at some point at infinity. Likewise, if
two lines aren't parallel, they'll have their own points at infinity,
but they won't be the same, thus they will only have the usual intersection
normally expected of non-parallel lines.

This may seem very strange, but can be illustrated with a stereographic
projection. Suppose we're in $\mathbb{R}^3$, and picture two things:
the plane $z=1$ and a sphere of radius 1 centered at the origin.
Now, we can map any point $p$ on $z=1$ to a unique point on the upper
hemisphere of our sphere; simply note that the line passing through
$p$ and the origin $(0, 0, 0)$ intersects the specified hemisphere
precisely once at some point $q$. Knowing this, we can see that lines
on our plane $z=1$ are mapped to great circles on our sphere. In the
case of two parallel lines, they will never intersect in $z=1$, but
the great circles they're mapped to will approach a point of intersection
on the equator of the hemisphere.

\begin{mydef}
A function $f(x)$ is \emph{Homogenous of Degree n} if, for some 
constant $\lambda$, we have $f(\lambda x) = \lambda ^n f(x)$.
If $f$ is a homogenous polynomial of degree n, it is known as a \emph{form of degree n}.
\end{mydef}

Homogenous functions are particularly useful when considering their
sets of zeroes in projective space. If $f$ is homogenous of degree
n, and some point $x = (x_0, ..., x_n)$ is a zero of $f$, then
$f(\lambda x) = \lambda^n f(x) = \lambda^n 0 = 0$. Thus, we can say
that, if $f$ is homogenous, then we can indeed have zeroes of $f$ in $\mathbb{P}^n_k$,
assuming k is algebraically closed.

\begin{mydef}
\[
S_d = \{\text{ forms of degree d in $(X, Y, Z)$}\}
\]
So, any element $F$ of $S_d$ can be written in a unique way as:
\[
F = \Sigma _{\substack{i+j+k'=d \\ i,j,k' > 0}}~a_{i,j,k'} X^i Y^j Z^{k'},~a_{i,j,k'} \in k
\]
\end{mydef}
So, $S_d$ has a basis of 
%improve formatting later...
\begin{align*}
\{&X^d, X^{d-1}Y, X^{d-2}Y^2, \dots\\
&X^{d-1}Z, X^{d-2}YZ, X^{d-3}Y^2Z, \dots\\
&X^{d-2}Z^2, X^{d-3}YZ^2, X^{d-4}Y^2Z^2, \dots\\
\vdots\\
&XZ^{d-1},YZ^{d-1}\\
&Z^d\}
\end{align*}
Also, it can be seen that $\dim S_d = {d+2 \choose 2}$. On its own
$S_d$ isn't incredibly interesting, but we can make use of it when
reasoning about certain sets of geometric objects.

\begin{mydef}
for $p_1, \dots, p_n \in \mathbb{P}^2$, let
\[
	S_d(p_1, \dots, p_n) = \{ F \in S_d :~F(p_i) = 0,~i=1,\dots,n\}
\]
\end{mydef}
Here, for example, we can augment $S_d$ with specific points to define
a set of all forms which are zero at those points. As we will see, zeroes of forms
define geometric figures, so reasoning about forms in these sets
is very useful for considering various kinds of intersections, an important
tool for various proofs.

\begin{mydef}
A \emph{line} in $\mathbb{P}^2$ is defined as the points for which
some linear form $aX + bY + cZ$ is zero. This is notated
\[
L:(aX+bY+cZ=0)
\]
\end{mydef}

As an example, consider $S_1$, the set of all linear forms in $\mathbb{R}^3$.
If $F \in S_1$ then $F = aX + bY + cZ$, so consider $L:(F=0)$, and examine
some point $p_0 = (X_0,Y_0,Z_0) \in L$ with $X_0 \ne 0, Y_0 \ne 0, Z_0 \ne 0$.
$p_0 \in L$ so $F(p_0) = aX_0 + bY_0 + cZ_0 = 0$, which
is a linear equation in 3 variables. None of the coefficients of $p_0$ are zero,
so we can easily determine one of our coefficients is dependent
on the others, thus, $S_1(p_0) \cong \mathbb{R}^2$, hence $\dim S_1(p_0) = 2$.

\begin{mydef}
A \emph{Projective Transformation} on $\mathbb{P}^2_\mathbb{R}$ is a
transformation of the form
\[
T(x) = Mx
\]
where $x$ is some point in $\mathbb{P}^2_\mathbb{R}$ in homogenous coordinates in vector form,
and $M$ is a non-singular $3\times3$ matrix.
\end{mydef}

More explicitly, if we restrict ourselves to $\mathbb{R}^2 \subset \mathbb{P}^2_\mathbb{R}$
(eg: the $Z=1$ plane) then we can think of this as
\[
\begin{bmatrix}
x\\
y
\end{bmatrix} \mapsto
(A\begin{bmatrix}
x\\
y
\end{bmatrix} + B) / (cx+dy+e)
\]
when $cx+dy+e \ne 0$. Thus, we can define the matrix $M$ as follows:
\[
M= \left[\begin{array}{cc|c}
\multicolumn{2}{c|}{A} & B\\
\hline
c & d & e
\end{array}\right]
\]

Interestingly, this is a somewhat subtle way of demonstrating that certain
non-linear transformations (projective and affine translation) in $n$
dimensions can actually be achieved using linear transformations in $n+1$ dimensions.
