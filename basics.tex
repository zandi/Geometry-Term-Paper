\section{Basics}
Here's where I'll put all the theorems and definitions that
are simply necessary. I'll definitely talk about projective
space here, and homogenous equations.

\begin{mydef}
Given a Field $k$, the $\emph{Projective Space of Degree n over k}$, denoted $\mathbb{P}^n_k$
is the quotient $(k^{n+1} \setminus \{\bold{0}\}) / \sim$, where $\sim$ is the equivalence relation 
defined as 
\[
(x_0, ..., x_n) \sim (y_0, ..., y_n)~if~\exists \lambda \in k~s.t.~\lambda \neq 0,
(x_0, ..., x_n) = (\lambda y_0, ..., \lambda y_n)
\]
Given this, points in projective space are often denoted $P = [ x_0 : ... : x_n]$,
with the colons and brackets indicating this is an equivalence class. So then,
$[ y_0: ... : y_n ] = P = [ x_0 : ... : x_n ]$ if $(y_0, ..., y_n) \sim (x_0, ..., x_n)$.
\end{mydef}
The definition is abstract to purposely highlight the fact that, 
given a field $k$ we can create a projective space.
Effectively, the projective space is the set of all lines in 
$k^n$ through the origin. This is also why the projective space
has one fewer dimension than its corresponding vector space.

So, the ``points" in projective space are lines in the corresponding
vector space. Another interesting feature of projective space is
that parallel lines intersect at a ``point at infinity". This can
be thought of as making partitions of lines in the projective space
pased on parallelism, as if we were defining an equivalence relation.
Now, each equivalence class is assigned a unique ``point at infinity"
that each line in the equivalence class intersects. Thus, if two lines
are parallel, they intersect at some point at infinity. Likewise, if
two lines aren't parallel, they'll have their own points at infinity,
but they won't be the same, thus they will only have the usual intersection
normally expected of non-parallel lines.

This may seem very strange, but can be illustrated with a stereographic
projection. Suppose we're in $\mathbb{R}^3$, and picture two things:
the plane $z=1$ and a sphere of radius 1 centered at the origin.
Now, we can map any point $p$ on $z=1$ to a unique point on the upper
hemisphere of our sphere; simply note that the line passing through
$p$ and the origin $(0, 0, 0)$ intersects the specified hemisphere
precisely once at some point $q$. Knowing this, we can see that lines
on our plane $z=1$ are mapped to great circles on our sphere. In the
case of two parallel lines, they will never intersect in $z=1$, but
the great circles they're mapped to will intersect on the equator of 
the hemisphere.

\begin{mydef}
A function $f(x)$ is \emph{Homogenous of Degree n} if, for some 
constant $\lambda$, we have $f(\lambda x) = \lambda ^n f(x)$.
If $f$ is a homogenous polynomial of degree n, it is known as a \emph{form of degree n}.
\end{mydef}

Homogenous functions are particularly useful when considering their
sets of zeroes in projective space. If $f$ is homogenous of degree
n, and some point $x = (x_0, ..., x_n)$ is a zero of $f$, then
$f(\lambda x) = \lambda^n f(x) = \lambda^n 0 = 0$. Thus, we can say
that, if $f$ is homogenous, then we can indeed have zeroes of $f$ in $\mathbb{P}^n_k$,
assuming k is algebraically closed.

\begin{mydef}
\[
S_d = \{\text{ forms of degree d in $(X, Y, Z)$}\}
\]
So, any element $F$ of $S_d$ can be written in a unique way as:
\[
F = \Sigma _{\substack{i+j+k'=d \\ i,j,k' > 0}}~a_{i,j,k'} X^i Y^j Z^{k'},~a_{i,j,k'} \in k
\]
\end{mydef}
So, $S_d$ has a basis of 
%improve formatting later...
\begin{align*}
\{&X^d, X^{d-1}Y, X^{d-2}Y^2, \dots\\
&X^{d-1}Z, X^{d-2}YZ, X^{d-3}Y^2Z, \dots\\
&X^{d-2}Z^2, X^{d-3}YZ^2, X^{d-4}Y^2Z^2, \dots\}
\end{align*}
Also, it can be seen that $\dim S_d = {d+2 \choose 2}$. On its own
$S_d$ isn't incredibly interesting, but we can make use of it when
reasoning about certain sets of geometric objects.

\begin{mydef}
for $p_1, \dots, p_n \in \mathbb{P}^2$, let
\[
	S_d(p_1, \dots, p_n) = \{ F \in S_d :~F(p_i) = 0,~i=1,\dots,n\}
\]
\end{mydef}
