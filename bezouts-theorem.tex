\section{Bezout's Theorem}
Bezout's theorem will be important later, but we
probably won't be able to fully prove it here. However,
the book does prove the limited cases that we'll most rely on.

\begin{theorem}
\emph{Bezout's Theorem}:\\

If $C$ and $D$ are plane curves of degree $m$ and $n$ (respectively)
then $C \cap D$ contains precisely $mn$ points, provided:
\begin{align*}
&\text{The field we're working over is algebraically closed} \tag{i}\\
&\text{Points of intersection are counted with multiplicity} \tag{ii}\\
&\text{We work in $\mathbb{P}^2_k$ to account for points at infinity} \tag{iii}\\
\end{align*}
\end{theorem}

Instead of proving the general theorem, we'll only cover a special 
case of particular importance to us; intersections involving
lines and conics.

\begin{theorem}
Let $L \subset \mathbb{P}^2_k$ be a line (respectively, $C \subset \mathbb{P}^2_K$ a non-degenerate conic)
and $D \subset \mathbb{P}^2_k$ a curve defined by $D:(G_d(X,Y,Z)=0)$ where $G$ is
a form of degree $d$ in the 3 variables $X,Y,Z$. Also, assume that $L \not\subset D$ (resp. $C \not\subset D$).
Then
\[
|L \cap D| \le d
\]
and respectively
\[
|C \cap D| \le 2d
\]
And in fact, once accounting for multiplicity, points at infinity, and if $k$ is algebraically
closed, we have equality.
\end{theorem}

\begin{proof}
The line $L$ is defined as the set of zeroes of some linear form $\lambda$; $L:(\lambda=0)$
in $X,Y,Z$. Well, as a line we can certainly paramaterize $\lambda$ using 
\[
X=a(U,V),~Y=b(U,V),~Z=c(U,V)
\]
where $a, b, c$ are linear forms of $U,V$. As an example, if $\lambda = \alpha X + \beta Y + \gamma Z$
with $\gamma \ne 0$ then we could paramaterize $L$ as
\[
X = U,~Y=V,~Z=-\frac{\alpha}{\gamma}U - \frac{\beta}{\gamma}V
\]
by simply noting that $\lambda = 0 \implies 0 = \alpha X + \beta Y + \gamma Z$ and solving
for $Z$. If $\gamma = 0$ we have the simpler case of $0 = \alpha X + \beta Y$.
Similarly, a non-degenerate conic can be parameterized as follows
\[
X=a(U,V),~Y=b(U,V),~Z=c(U,V)
\]
where $a,b,c$ are quadratic forms in $U,V$. This is because $C$ is a projective
transformation of $XZ=Y^2$, which is parametrically $(X,Y,Z) = (U^2,UV,V^2)$.
Thus, $C$ is given by 
\[
\begin{bmatrix}
X\\
Y\\
Z
\end{bmatrix}
=
M
\begin{bmatrix}
U^2\\
UV\\
V^2
\end{bmatrix}
\]
Where $M$ is a non-singular 3x3 matrix.
\end{proof}
