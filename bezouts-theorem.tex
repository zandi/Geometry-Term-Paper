\section{Bezout's Theorem}

One of the most beautiful and powerful theorems in
algebraic geometry is Bezout's Theorem. With it we
can very precisely link the intersections of 
geometric structures with the degree of their
defining forms, and characterize their intersection.
This is quite useful in proofs, as various kinds
of intersections occur quite frequently. To start,
we'll explore some of the properties of homogenous
forms in 2 variables, and the explicit connection
between these forms and their associated non-homogenous 
polynomials. It may be useful here to consider
the plane $Z=1$ when in $\mathbb{P}^2$, and the line
$V=1$ when in $\mathbb{P}^1$.

\begin{theorem}
\emph{Homogenous Forms in 2 Variables}:\\

Let $F(U,V)$ be a non-zero homogenous polynomial of degree $d$ in $U,V$,
with coefficients from a fixed field $k$. Then, we call $F$ a \emph{form of degree d},
and can see that
\[
F(U,V) = a_dU^d+a_{d-1}U^{d-1}V + \dots + a_iU^iV^{d-i}+\dots+a_0V^d
\]
Now, if we apply $u=U/V$ and hold $V=1$ we have an associated non-homogenous polynomial
\[
f(u) = a_du^d + a_{d-1}u^{d-1} + \dots + a_iu^i + \dots + a_0
\]
\end{theorem}


This may not seem incredibly important, but this observation
allows us to explore the zeroes of forms by examining the zeroes
of relatively simpler polynomials. Similar constructions can be made
for forms in more variables, but this two-variable form gives us a relatively
simple example to explore.\\

Exploring this further, for some $\alpha \in k$
\[ %a for k-valued root, alpha for... something in P^1?
f(a) = 0 \iff (u-a)|f(u) \iff (U-\alpha V) | F(U,V)
\]
As $f$ must contain at least a linear term whose root is $a$, and
thus applying $u=U/V$, we get $(U/V - a)$ and thus $(U - \alpha V)$. Now, 
holding $V=1$ like earlier, we see that
\[
(U-\alpha V) | F(U,V) \iff F(\alpha,1) = 1
\]
So, zeroes of the simpler polynomial $f$ correspond to zeroes of $F$
on $\mathbb{P}^1_k$ so long as $V \ne 0$. However, if $F$ had a root
of the form $(\beta,0)$ then certainly by homogeneity $F$ also has
$(1,0)$ as a root, giving:
\[
F(1,0) = 0 \iff a_d \cdot 1 + a_{d-1} \cdot 1 \cdot 0 + \dots + a_0 \cdot 0 = 0 \iff a_d = 0
\]
which would force $deg f < d$, as then the highest power of $U$ in $F$ is strictly
less than $d$. In this situation, $(1,0)$ is effectively a ``root at $\infty$" in $\mathbb{P}^1_k$.

Now, we can define the \emph{multiplicity} of a zero of $F$ as:
\begin{align*}
&\text{The multiplicity of $f$ at the corresponding $a \in k$} \tag{i}\\
&\text{$d - deg f$ if $(1,0)$ is the zero} \tag{ii}
\end{align*}
Equivalently, the multiplicity of a zero of $F$ at a point $(a,1)$
is the greatest power of $(U - \alpha V)$ dividing $F$, and at $(1,0)$
is the greatest power of $V$ dividing $F$.

So, we have an interesting relationship between forms of 2 variables and their
related polynomials in a single variable. Assuming $k$ is algebraically closed, we are in fact
bridging the zeroes of $F$ in $\mathbb{P}^1_k$ with the zeroes of $f$
in $k$ by examining the restriction of $F$ to the line $V=1$.
Later we'll see a similar approach where third-degree forms with roots
$(X,Y,Z) \in \mathbb{P}^2_k$ are examined by restricting $F$ to the
$Z=1$ plane, giving us some simpler $f$ with roots in $k^2$.

\begin{theorem}
\emph{Bezout's Theorem}:\\

If $C$ and $D$ are plane curves of degree $m$ and $n$ (respectively)
then $C \cap D$ contains precisely $mn$ points, provided:
\begin{align*}
&\text{The field we're working over is algebraically closed} \tag{i}\\
&\text{Points of intersection are counted with multiplicity} \tag{ii}\\
&\text{We work in $\mathbb{P}^2_k$ to account for points at infinity} \tag{iii}\\
\end{align*}
\end{theorem}

Instead of proving the general theorem, we'll only cover a special 
case of particular importance to us; intersections involving
lines and conics.

\begin{theorem}
Let $L \subset \mathbb{P}^2_k$ be a line (respectively, $C \subset \mathbb{P}^2_K$ a non-degenerate conic)
and $D \subset \mathbb{P}^2_k$ a curve defined by $D:(G_d(X,Y,Z)=0)$ where $G$ is
a form of degree $d$ in the 3 variables $X,Y,Z$. Also, assume that $L \not\subset D$ (resp. $C \not\subset D$).
Then
\[
|L \cap D| \le d
\]
and respectively
\[
|C \cap D| \le 2d
\]
And in fact, once accounting for multiplicity, points at infinity, and if $k$ is algebraically
closed, we have equality.
\end{theorem}

\begin{proof}
The line $L$ is defined as the set of zeroes of some linear form $\lambda$; $L:(\lambda=0)$
in $X,Y,Z$. Well, as a line we can certainly paramaterize $L$ using 
\[
X=a(U,V),~Y=b(U,V),~Z=c(U,V)
\]
where $a, b, c$ are linear forms of $U,V$. As an example, if $\lambda = \alpha X + \beta Y + \gamma Z$
with $\gamma \ne 0$ then we could paramaterize $L$ as
\[
X = U,~Y=V,~Z=-\frac{\alpha}{\gamma}U - \frac{\beta}{\gamma}V
\]
by simply noting that $\lambda = 0 \implies 0 = \alpha X + \beta Y + \gamma Z$ and solving
for $Z$. If $\gamma = 0$ we have the simpler case of $0 = \alpha X + \beta Y$.
Similarly, a non-degenerate conic can be parameterized as follows
\[
X=a(U,V),~Y=b(U,V),~Z=c(U,V)
\]
where $a,b,c$ are quadratic forms in $U,V$. This is because $C$ is a projective
transformation of $XZ=Y^2$, which is can be parameterized by $(X,Y,Z) = (U^2,UV,V^2)$.
Thus, $C$ is given by 
\[
\begin{bmatrix}
X\\
Y\\
Z
\end{bmatrix}
=
M
\begin{bmatrix}
U^2\\
UV\\
V^2
\end{bmatrix}
\]
Where $M$ is a non-singular 3x3 matrix. Hence, $C$ can be parameterized by
quadratic forms $a, b, c$ in $U,V$.

So, in order to find the points of intersection of $L$ (resp. $C$) with $D$,
we simply need to evaluate $G_d|_L=0$ (resp. $G_d|_C=0$) using the above parameterizations.
Explicitly, to find the points of $L \cap D$ we want to find the solutions to
\[
	G_d(a(U,V),b(U,V),c(U,V)) = 0
\]
Now we can see that, as $G_d$ is a form of degree $d$ in $X,Y,Z$, and $a,b,c$ are forms
of degree 1 (resp. 2) in $U,V$ then by composition $G_d|_L$ is of degree $d$ (resp. 2d) in $U,V$.

Now, if $k$ is algebraically closed, then $G_d \in k[X,Y,Z]$ and so all roots of $G_d|_L$ are
elements $(U, V)$ of $k \times k$. Likewise, all roots of $G_d|_C$ are elements of $k \times k$.
Thus, accounting for multiplicities and points at infinity, $G_d|_L$ has precisely $d$ roots in 
$(U, V)$ and $G_d|_C$ has precisely $2d$ roots in $(U, V)$.
Using the parameterizations $a, b, c$ we can easily find the corresponding
roots in $\mathbb{P}^2_k$.

Thus, accounting for multiplicities of zeroes and points at infinity,
the intersection $L \cap D$ has $d$ elements, and the intersection
$C \cap D$ has $2d$ elements. Thus, the curve $L$ (resp. $C$) meets $D$ in
precisely $d$ (resp. $2d$) points.
\end{proof}

Now, we formally know that given two distinct lines in $\mathbb{P}^2$, they will intersect in
precisely one point. Likewise, the intersection of a line with a conic that doesn't
contain it will have precisely $2$ points. This is a useful tool when combined with
examining the dimension of $S_d$. For example, the intersection of any two
conics $C_1,~C_2$ has $4$ points, so given $5$ points, what do
we know about the conics which lie on them? If no $4$ of these $5$ points
are collinear, then in fact there is only one conic $C$ which passes
through all of them. This can be shown by assuming the contrary;
that two distinct conics pass through such $5$ points; and then examining
the intersection of these conics. Another interesting consequence
is that, with distinct $p_0, \dots, p_n \in \mathbb{P}^2_\mathbb{R}$, then
$\dim S_2(p_0, \dots, p_n) \ge 6 - n$. This can be seen by considering
some $Q \in S_2(p_0, \dots, p_n)$, and treating each of $Q(p_i) = 0$ as
a linear equation in $6$ coefficients, since each $p_i$ is fixed. Then we 
have a system of $n$ equations in $6$ variables, and we can easily apply
linear algebra techniques.
