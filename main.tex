\documentclass[draft]{article}
\usepackage{amssymb,amsmath,amsthm,amsfonts}
\renewcommand{\abstractname}{Introduction} %change title of the abstract section
\newtheorem{mydef}{Definition}[section]
\newtheorem{lemma}{Lemma}[section]
\newtheorem{theorem}{Theorem}[section]

\begin{document}


\title{Algebraic Geometry}
\author{Zandi} %will use full information for assignment submission
\date{\today}
\maketitle

%TODO: remove me when finished
THIS IS A DRAFT. Expect inaccuracte and incomplete information.

\begin{abstract}
Algebraic Geometry is a field of mathematics with roots as far back as
at least the 19\textsuperscript{th} century, and with major advancement and applications
in recent years. At its core, Algebraic Geometry is concerned with 
studying the sets of zeroes of a system of polynomials, in particular 
their algebraic structure. Though real polynomials are surely the first
that come to mind, much of the power in Algebraic Geometry comes from the
generalization to polynomials with coefficients in arbitrary fields.\\

The undergraduate mathematics student would hopefully find this paper accessible,
but experience with algebra and geometry would help. A ``Basics" section is included,
but will not be exhaustive. As unhelpful as it seems, it's recommended to 
do independent research as necessary using Wikipedia, YouTube (videotaped lectures)
and the referenced textbooks.

This paper draws heavily from ``Undergraduate Algebraic Geometry" by Miles Reid, 
with the primary addition being annotated proofs. Where Reid's proofs
left details to the reader, these details are mostly provided, although
the proofs remain the same.
\end{abstract}

%\section{Basics}
Here's where I'll put all the theorems and definitions that
are simply necessary. I'll definitely talk about projective
space here, and homogenous equations.

\begin{mydef}
Given a Field $k$, the $\emph{Projective Space of Degree n over k}$, denoted $\mathbb{P}^n_k$
is the quotient $(k^{n+1} \setminus \{\bold{0}\}) / \sim$, where $\sim$ is the equivalence relation 
defined as 
\[
(x_0, ..., x_n) \sim (y_0, ..., y_n)~if~\exists \lambda \in k~s.t.~\lambda \neq 0,
(x_0, ..., x_n) = (\lambda y_0, ..., \lambda y_n)
\]
Given this, points in projective space are often denoted $P = [ x_0 : ... : x_n]$,
with the colons and brackets indicating this is an equivalence class. So then,
$[ y_0: ... : y_n ] = P = [ x_0 : ... : x_n ]$ if $(y_0, ..., y_n) \sim (x_0, ..., x_n)$.
\end{mydef}
The definition is abstract to purposely highlight the fact that, 
given a field $k$ we can create a projective space.
Effectively, the projective space is the set of all lines in 
$k^n$ through the origin. This is also why the projective space
has one fewer dimension than its corresponding vector space.

So, the ``points" in projective space are lines in the corresponding
vector space. Another interesting feature of projective space is
that parallel lines intersect at a ``point at infinity". This can
be thought of as making partitions of lines in the projective space
pased on parallelism, as if we were defining an equivalence relation.
Now, each equivalence class is assigned a unique ``point at infinity"
that each line in the equivalence class intersects. Thus, if two lines
are parallel, they intersect at some point at infinity. Likewise, if
two lines aren't parallel, they'll have their own points at infinity,
but they won't be the same, thus they will only have the usual intersection
normally expected of non-parallel lines.

This may seem very strange, but can be illustrated with a stereographic
projection. Suppose we're in $\mathbb{R}^3$, and picture two things:
the plane $z=1$ and a sphere of radius 1 centered at the origin.
Now, we can map any point $p$ on $z=1$ to a unique point on the upper
hemisphere of our sphere; simply note that the line passing through
$p$ and the origin $(0, 0, 0)$ intersects the specified hemisphere
precisely once at some point $q$. Knowing this, we can see that lines
on our plane $z=1$ are mapped to great circles on our sphere. In the
case of two parallel lines, they will never intersect in $z=1$, but
the great circles they're mapped to will intersect on the equator of 
the hemisphere.

\begin{mydef}
A function $f(x)$ is \emph{Homogenous of Degree n} if, for some 
constant $\lambda$, we have $f(\lambda x) = \lambda ^n f(x)$.
If $f$ is a homogenous polynomial of degree n, it is known as a \emph{form of degree n}.
\end{mydef}

Homogenous functions are particularly useful when considering their
sets of zeroes in projective space. If $f$ is homogenous of degree
n, and some point $x = (x_0, ..., x_n)$ is a zero of $f$, then
$f(\lambda x) = \lambda^n f(x) = \lambda^n 0 = 0$. Thus, we can say
that, if $f$ is homogenous, then we can indeed have zeroes of $f$ in $\mathbb{P}^n_k$,
assuming k is algebraically closed.

%this seemed like a good definition, but is overly broad. need to find 
%a more precise definition.
%\begin{mydef}
%For the purposes of this paper, a \emph{line} is a set whose points can be parameterised in terms
%of a single variable. That is, for some subset $l$ of our space, there exist functions $f_n$ s.t.
%\[
%	l = \{x : x = (x_1, x_2, \dots, x_n)\}
%\]
%where
%\[
%	(x_1, x_2, \dots, x_n) = (f_1(t), f_2(t), \dots, f_n(t))
%\]
%\end{mydef}
%
%As an illustrative example, in $\mathbb{R}^2$ we know that 
%$l = \{(x,mx+b):x\in\mathbb{R}\}$ is a line, and we can see that
%$l$ is fully parameterized in terms of $x$. Likewise, in $\mathbb{R}^n$,
%if we're given two points $a, b \in \mathbb{R}^n$ we can uniquely
%determine the line $l$ s.t. $a \in l, b \in l$ as follows:
%\[
%	l = \{ (1-t)a + tb : t \in \mathbb{R}\}
%\]
%This may cause some initial confusion when considering lines in projective
%spaces, such as $\mathbb{P}^2$. However, simply remember that the points in
%$\mathbb{P}^2$ are equivalence classes of some affine space of dimension $3$,
%such as $\mathbb{R}^3$ or $k^3$. Using a restriction such as $Z=1$ for
%$(X,Y,Z) \in \mathbb{P}^2$ can help with this.
%
%Of particular importance is that, when in $\mathbb{P}^2_k$, all lines will have
%some linear form whose set of roots determines the line.

\begin{mydef}
\[
S_d = \{\text{ forms of degree d in $(X, Y, Z)$}\}
\]
So, any element $F$ of $S_d$ can be written in a unique way as:
\[
F = \Sigma _{\substack{i+j+k'=d \\ i,j,k' > 0}}~a_{i,j,k'} X^i Y^j Z^{k'},~a_{i,j,k'} \in k
\]
\end{mydef}
So, $S_d$ has a basis of 
%improve formatting later...
\begin{align*}
\{&X^d, X^{d-1}Y, X^{d-2}Y^2, \dots\\
&X^{d-1}Z, X^{d-2}YZ, X^{d-3}Y^2Z, \dots\\
&X^{d-2}Z^2, X^{d-3}YZ^2, X^{d-4}Y^2Z^2, \dots\}
\end{align*}
Also, it can be seen that $\dim S_d = {d+2 \choose 2}$. On its own
$S_d$ isn't incredibly interesting, but we can make use of it when
reasoning about certain sets of geometric objects.

\begin{mydef}
for $p_1, \dots, p_n \in \mathbb{P}^2$, let
\[
	S_d(p_1, \dots, p_n) = \{ F \in S_d :~F(p_i) = 0,~i=1,\dots,n\}
\]
\end{mydef}

\section{Basics}
Here's where I'll put all the theorems and definitions that
are simply necessary. I'll definitely talk about projective
space here, and homogenous equations.

\begin{mydef}
Given a Field $k$, the $\emph{Projective Space of Degree n over k}$, denoted $\mathbb{P}^n_k$
is the quotient $(k^{n+1} \setminus \{\bold{0}\}) / \sim$, where $\sim$ is the equivalence relation 
defined as 
\[
(x_0, ..., x_n) \sim (y_0, ..., y_n)~if~\exists \lambda \in k~s.t.~\lambda \neq 0,
(x_0, ..., x_n) = (\lambda y_0, ..., \lambda y_n)
\]
Given this, points in projective space are often denoted $P = [ x_0 : ... : x_n]$,
with the colons and brackets indicating this is an equivalence class. So then,
$[ y_0: ... : y_n ] = P = [ x_0 : ... : x_n ]$ if $(y_0, ..., y_n) \sim (x_0, ..., x_n)$.
\end{mydef}
The definition is abstract to purposely highlight the fact that, 
given a field $k$ we can create a projective space.
Effectively, the projective space is the set of all lines in 
$k^n$ through the origin. This is also why the projective space
has one fewer dimension than its corresponding vector space.

So, the ``points" in projective space are lines in the corresponding
vector space. Another interesting feature of projective space is
that parallel lines intersect at a ``point at infinity". This can
be thought of as making partitions of lines in the projective space
pased on parallelism, as if we were defining an equivalence relation.
Now, each equivalence class is assigned a unique ``point at infinity"
that each line in the equivalence class intersects. Thus, if two lines
are parallel, they intersect at some point at infinity. Likewise, if
two lines aren't parallel, they'll have their own points at infinity,
but they won't be the same, thus they will only have the usual intersection
normally expected of non-parallel lines.

This may seem very strange, but can be illustrated with a stereographic
projection. Suppose we're in $\mathbb{R}^3$, and picture two things:
the plane $z=1$ and a sphere of radius 1 centered at the origin.
Now, we can map any point $p$ on $z=1$ to a unique point on the upper
hemisphere of our sphere; simply note that the line passing through
$p$ and the origin $(0, 0, 0)$ intersects the specified hemisphere
precisely once at some point $q$. Knowing this, we can see that lines
on our plane $z=1$ are mapped to great circles on our sphere. In the
case of two parallel lines, they will never intersect in $z=1$, but
the great circles they're mapped to will intersect on the equator of 
the hemisphere.

\begin{mydef}
A function $f(x)$ is \emph{Homogenous of Degree n} if, for some 
constant $\lambda$, we have $f(\lambda x) = \lambda ^n f(x)$.
If $f$ is a homogenous polynomial of degree n, it is known as a \emph{form of degree n}.
\end{mydef}

Homogenous functions are particularly useful when considering their
sets of zeroes in projective space. If $f$ is homogenous of degree
n, and some point $x = (x_0, ..., x_n)$ is a zero of $f$, then
$f(\lambda x) = \lambda^n f(x) = \lambda^n 0 = 0$. Thus, we can say
that, if $f$ is homogenous, then we can indeed have zeroes of $f$ in $\mathbb{P}^n_k$,
assuming k is algebraically closed.

%this seemed like a good definition, but is overly broad. need to find 
%a more precise definition.
%\begin{mydef}
%For the purposes of this paper, a \emph{line} is a set whose points can be parameterised in terms
%of a single variable. That is, for some subset $l$ of our space, there exist functions $f_n$ s.t.
%\[
%	l = \{x : x = (x_1, x_2, \dots, x_n)\}
%\]
%where
%\[
%	(x_1, x_2, \dots, x_n) = (f_1(t), f_2(t), \dots, f_n(t))
%\]
%\end{mydef}
%
%As an illustrative example, in $\mathbb{R}^2$ we know that 
%$l = \{(x,mx+b):x\in\mathbb{R}\}$ is a line, and we can see that
%$l$ is fully parameterized in terms of $x$. Likewise, in $\mathbb{R}^n$,
%if we're given two points $a, b \in \mathbb{R}^n$ we can uniquely
%determine the line $l$ s.t. $a \in l, b \in l$ as follows:
%\[
%	l = \{ (1-t)a + tb : t \in \mathbb{R}\}
%\]
%This may cause some initial confusion when considering lines in projective
%spaces, such as $\mathbb{P}^2$. However, simply remember that the points in
%$\mathbb{P}^2$ are equivalence classes of some affine space of dimension $3$,
%such as $\mathbb{R}^3$ or $k^3$. Using a restriction such as $Z=1$ for
%$(X,Y,Z) \in \mathbb{P}^2$ can help with this.
%
%Of particular importance is that, when in $\mathbb{P}^2_k$, all lines will have
%some linear form whose set of roots determines the line.

\begin{mydef}
\[
S_d = \{\text{ forms of degree d in $(X, Y, Z)$}\}
\]
So, any element $F$ of $S_d$ can be written in a unique way as:
\[
F = \Sigma _{\substack{i+j+k'=d \\ i,j,k' > 0}}~a_{i,j,k'} X^i Y^j Z^{k'},~a_{i,j,k'} \in k
\]
\end{mydef}
So, $S_d$ has a basis of 
%improve formatting later...
\begin{align*}
\{&X^d, X^{d-1}Y, X^{d-2}Y^2, \dots\\
&X^{d-1}Z, X^{d-2}YZ, X^{d-3}Y^2Z, \dots\\
&X^{d-2}Z^2, X^{d-3}YZ^2, X^{d-4}Y^2Z^2, \dots\}
\end{align*}
Also, it can be seen that $\dim S_d = {d+2 \choose 2}$. On its own
$S_d$ isn't incredibly interesting, but we can make use of it when
reasoning about certain sets of geometric objects.

\begin{mydef}
for $p_1, \dots, p_n \in \mathbb{P}^2$, let
\[
	S_d(p_1, \dots, p_n) = \{ F \in S_d :~F(p_i) = 0,~i=1,\dots,n\}
\]
\end{mydef}

\section{Conic Sections}

Conic sections provide perhaps the simplest geometric
figures to study in algebraic geometry. They are fairly
well understood within $\mathbb{R}^2$, and extend quite beautifully
into $\mathbb{P}^2$. One example of this is examining 
the classification of conics in both $\mathbb{R}^2$ and
$\mathbb{P}^2$. While both have their degenerate cases,
it's interesting to note that in $\mathbb{R}^2$ there
are 3 non-degenerate cases of a conic section;
the ellipse, hyperbola, and parabola. However, in
$\mathbb{P}^2$ these can all be described in a single
non-degenerate case, thanks to the point at infinity of
projective space. Also, algebraic geometry can inform
the historically interesting execercise of finding
rational or integer roots for the pythagorean equation
$X^2 + Y^2 = Z^2$.

However, we will only briefly cover conic sections, and even
then primarily for their application to forthcoming sections.
Let's begin by defining what a conic actually is.

\begin{mydef}
A \emph{Conic} in $\mathbb{R}^2$ is a plane curve given by the quadratic
equation
\[
q(x,y) = ax^2 + bxy + cy^2 + dx + ey + f = 0
\]
\end{mydef}

To ``adapt" this definition into projective space $\mathbb{P}^2$, we simply
note that the inhomogenous polynomial $q$ has an associated
homogenous polynomial 
\[
Q(X,Y,Z) = aX^2 + bXY + cY^2 + dXZ + eYZ + fZ^2
\]
Miles Reid recommends thinking about this as a bijection $q \leftrightarrow Q$ by
\[
q(x,y) = Q(\frac{X}{Z},\frac{Y}{Z},\frac{Z}{Z}),~Z\ne 0
\]
Thus, a conic in $\mathbb{P}^2$ would be determined by the zeroes of $Q$.

\begin{theorem}
Let $k$ be any field of characteristic $\ne 2$. Then, any quadratic form $Q$
in $(X,Y,Z)$ has an associated $3 \times 3$ symmetric matrix $A$ s.t.
\[
x^TAx =  Q(X,Y,Z),~x=\begin{bmatrix}
x\\
y\\
z
\end{bmatrix}
\]
explicitly, if 
\[
Q(X,Y,Z) = aX^2 + 2bXY + cY^2 + 2dXZ + 2eYZ + fZ^2
\]
Then
\[
A = \begin{bmatrix}
a & b & d\\
b & c & e\\
d & e & f
\end{bmatrix}
\]
and the quadratic form (and hence the conic it defines) is non-degenerate
if the matrix $A$ is non-degenerate (non-singular).
\end{theorem}

Note that the matrix $A$ from the above theorem is symmetric. Thus, we can
apply the Spectral Theorem to see that we can diagonalize it using some
orthogonal matrix. This effectively allows us to transform any non-degenerate 
quadratic form into a homogenous quadratic form through a change of basis.

\begin{theorem}
Let $Q$ be a quadratic form of $X, Y, Z$. Then, there exists a change of basis $(X,Y,Z) \mapsto (X',Y',Z')$ s.t.
$Q(X,Y,Z) = Q'(X',Y',Z')$ where
\[
Q'(X',Y',Z') = \lambda_1 X'^2 + \lambda_2 Y'^2 + \lambda_3 Z'^2
\]
As a corollary, all non-degenerate conics in $\mathbb{P}^2$ are projectively
equivalent to the conic defined by $XZ = Y^2$. That is, there is some
projective transformation which takes an arbitrary conic $C$ to the conic 
defined by $XZ = Y^2$.
\end{theorem}
\begin{proof}
$Q$ is a quadratic form,
thus $Q$ has an associated symmetric matrix 
$A$ s.t. $Q(x) = x^TAx$. As $A$ is symmetric, the Spectral Theorem gives that
\[
\exists P:~P\text{ orthogonal},~P^TAP\text{ is diagonal}
\]
and in fact, the diagonal entries of $B=P^TAP$ are the eigenvalues of $A$. Thus,
it is easy to show, using the transformation $x=Px'$, that
\[
Q(X,Y,Z) = x^TAx = (Px')^TAPx' = x'^TP^TAPx' = x'^TBx'
\]
Now, letting the eigenvalues of $A$ be $\lambda_1, \lambda_2, \lambda_3$ we see that
\[
x'^TBx' = \lambda_1 X'^2 + \lambda_2 Y'^2 + \lambda_3 Z'^2 = Q'(X',Y',Z')
\]
Thus, $Q(X,Y,Z) = Q'(X',Y',Z')$.
\end{proof}
Thus, conic sections in $\mathbb{P}^2$ have an interesting sort of 
unifying characteristic when regarded under projective transformations.
Not only is there one case of a non-degenerate conic instead of 3,
but any non-degenerate conic is ultimately just a projective transformation
of $XZ=Y^2$!. This will come in handy in just a few pages.

\section{Bezout's Theorem}
Bezout's theorem will be important later, but we
probably won't be able to fully prove it here. However,
the book does prove the limited cases that we'll most rely on.

\begin{theorem}
\emph{Homogenous Forms in 2 Variables}:\\

Let $F(U,V)$ be a non-zero homogenous polynomial of degree $d$ in $U,V$,
with coefficients from a fixed field $k$. Then, we call $F$ a \emph{form of degree d},
and can see that
\[
F(U,V) = a_dU^d+a_{d-1}U^{d-1}V + \dots + a_iU^iV^{d-i}+\dots+a_0V^d
\]
Now, if we apply $u=U/V$ and hold $V=1$ we have an associated non-homogenous polynomial
\[
f(u) = a_du^d + a_{d-1}u^{d-1} + \dots + a_iu^i + \dots + a_0
\]
And so, for some $\alpha \in k$
\[ %a for k-valued root, alpha for... something in P^1?
f(a) = 0 \iff (u-a)|f(u) \iff (U-\alpha V) | F(U,V)
\]
As $f$ must contain at least a linear term whose root is $a$, and
thus applying $u=U/V$, we get $(U/V - a) = (U - \alpha V)$. Now, 
holding $V=1$ like earlier, we see that
\[
(U-\alpha V) | F(U,V) \iff F(\alpha,1) = 1
\]
So, zeroes of the simpler polynomial $f$ correspond to zeroes of $F$
on $\mathbb{P}^1_k$ so long as $V \ne 0$. However, if $F$ had a root
of the form $(\beta,0)$ then certainly by homogeneity $F$ also has
$(1,0)$ as a root, giving:
\[
F(1,0) = 0 \iff a_d \cdot 1 + a_{d-1} \cdot 1 \cdot 0 + \dots + a_0 \cdot 0 = 0 \iff a_d = 0
\]
which would force $deg f < d$, as then the highest power of $U$ in $F$ is strictly
less than $d$. In this situation, $(1,0)$ is effectively a ``root at $\infty$" in $\mathbb{P}^1_k$.

Now, we can define the \emph{multiplicity} of a zero of $F$ as:
\begin{align*}
&\text{The multiplicity of $f$ at the corresponding $a \in k$}\\
&\text{$d - deg f$ if $(1,0)$ is the zero}
\end{align*}

\end{theorem}

\begin{theorem}
\emph{Bezout's Theorem}:\\

If $C$ and $D$ are plane curves of degree $m$ and $n$ (respectively)
then $C \cap D$ contains precisely $mn$ points, provided:
\begin{align*}
&\text{The field we're working over is algebraically closed} \tag{i}\\
&\text{Points of intersection are counted with multiplicity} \tag{ii}\\
&\text{We work in $\mathbb{P}^2_k$ to account for points at infinity} \tag{iii}\\
\end{align*}
\end{theorem}

Instead of proving the general theorem, we'll only cover a special 
case of particular importance to us; intersections involving
lines and conics.

\begin{theorem}
Let $L \subset \mathbb{P}^2_k$ be a line (respectively, $C \subset \mathbb{P}^2_K$ a non-degenerate conic)
and $D \subset \mathbb{P}^2_k$ a curve defined by $D:(G_d(X,Y,Z)=0)$ where $G$ is
a form of degree $d$ in the 3 variables $X,Y,Z$. Also, assume that $L \not\subset D$ (resp. $C \not\subset D$).
Then
\[
|L \cap D| \le d
\]
and respectively
\[
|C \cap D| \le 2d
\]
And in fact, once accounting for multiplicity, points at infinity, and if $k$ is algebraically
closed, we have equality.
\end{theorem}

\begin{proof}
The line $L$ is defined as the set of zeroes of some linear form $\lambda$; $L:(\lambda=0)$
in $X,Y,Z$. Well, as a line we can certainly paramaterize $L$ using 
\[
X=a(U,V),~Y=b(U,V),~Z=c(U,V)
\]
where $a, b, c$ are linear forms of $U,V$. As an example, if $\lambda = \alpha X + \beta Y + \gamma Z$
with $\gamma \ne 0$ then we could paramaterize $L$ as
\[
X = U,~Y=V,~Z=-\frac{\alpha}{\gamma}U - \frac{\beta}{\gamma}V
\]
by simply noting that $\lambda = 0 \implies 0 = \alpha X + \beta Y + \gamma Z$ and solving
for $Z$. If $\gamma = 0$ we have the simpler case of $0 = \alpha X + \beta Y$.
Similarly, a non-degenerate conic can be parameterized as follows
\[
X=a(U,V),~Y=b(U,V),~Z=c(U,V)
\]
where $a,b,c$ are quadratic forms in $U,V$. This is because $C$ is a projective
transformation of $XZ=Y^2$, which is parametrically $(X,Y,Z) = (U^2,UV,V^2)$.
Thus, $C$ is given by 
\[
\begin{bmatrix}
X\\
Y\\
Z
\end{bmatrix}
=
M
\begin{bmatrix}
U^2\\
UV\\
V^2
\end{bmatrix}
\]
Where $M$ is a non-singular 3x3 matrix. (JUSTIFY THIS!!!)

So, in order to find the points of intersection of $L$ (resp. $C$) with $D$,
we simply need to evaluate $G_d|_L=0$ (resp. $G_d|_C=0$) using the above parameterizations.
Explicitly, to find the points of $L \cap D$ we want to find the solutions to
\[
	G_d(a(U,V),b(U,V),c(U,V)) = 0
\]
Now we can see that, as $G_d$ is a form of degree $d$ in $X,Y,Z$, and $a,b,c$ are forms
of degree 1 (resp. 2) in $U,V$ then $G_d|_L$ is of degree $d$ (resp. 2d) in $U,V$.
\end{proof}

\section{Elliptic Curves}

One of the more interesting sections. Definitely need to
cover proving the given construction on the non-singuler elliptic
curve creates a group. Should also cover how in practice for crypto
the identity is chosen as the point at infinity, allowing for 
much cheaper computation.

Current strategy is to develop this section, then determine
dependencies for understanding, and stuff those in the
earlier sections.

\hrulefill

\begin{mydef}
\label{def:ellipticgroup}
\emph{Group Law on a Plane Cubic:}
Suppose $k \subset \mathbb{C}$ is a subfield of $\mathbb{C}$,
and $F \in k[X,Y,Z]$ is a cubic form defining a non-empty plane
curve $C:(F=0) \subset \mathbb{P}^2_k$. Additionally, assume that
$F$ satisfies the following:
\begin{align*}
F \text{ is irreducible, (C won't contain a line or conic)}&\tag{a}\\
\forall P \in C,~\exists! L \subset \mathbb{P}^2_k~s.t.~ P
\text{ is a repeated zero of } F|_L& \tag{b}\\
\end{align*}

Now, fix any point $O \in C$ and make the following construction:
\begin{align*}
&\text{for } A \in C, \text{ let } \bar{A} \text{ be the third point of intersection of $C$ with $\overleftrightarrow{OA}$} \tag{i}\\
&\text{for } A, B \in C \text{ let } R \text{ be the third point of intersection of $C$ with $\overleftrightarrow{AB}$}\\
&\text{ and define $A+B = \bar{R}$}\tag{ii}\\
\end{align*}
\end{mydef}

Geometrically, the uniqueness in (b) says that $C$ should be non-singular,
and we can see that this line $L$ is in fact the tangent line to $C$ at $P$.
Likewise, (a) prevents somewhat degenerate cases.

As you may have noticed, we've defined an operation on points of this
cubic curve $C$, so naturally we're curious what sort of structure then
arises. It turns out that this actually defines an Abelian Group on $C$,
as we will now prove.

\begin{theorem}
The above construction in Definition \ref{def:ellipticgroup} defines
an Abelian Group $(C,+)$ with $O$ as the identity element.
\end{theorem}

\begin{proof}
Well, to prove this is an abelian group, we have a few conditions we must satisfy:
\begin{enumerate}
\item Well-definedness of the operation and inverses
\item Existence of an identity element
\item Existence of unique inverses
\item Associativity
\end{enumerate}

Let $P, Q \in C$. Then, we have two simple cases:\\

case 1:  $P \ne Q$. Then $\exists! L \subset \mathbb{P}^2_k$ such that $L = \overleftrightarrow{PQ}$.\\

case 2: $P = Q$. Then, part (b) of the definition gives that $\exists! L \subset \mathbb{P}^2_k$ such that
$P$ is a repeated root of $F|_L$.

Note, however, that in both cases $F_L$ is a cubic form in 2 variables. $L$ is 
a line, and thus defined as the set of zeroes of some linear form $l \in k[X,Y,Z]$. As is common, we've
held $Z$ constant and non-zero to simplify our work, and use the common substitution
$x=X/Z,~y=Y/Z$. Thus, as $F|_L = F \circ l$, $F|_L$ can only be of 3rd degree,
and is still homogenous.

So, as $F|_L$ is a 3rd degree form, it has precisely 3 roots, 2 of which
are known to be $P$ and $Q$. Well, $P \in \mathbb{P}^2_k,~Q\in \mathbb{P}^2_k$
so $F|_L$ is the product of 3 linear terms from $k[X,Y,Z]$. Thus, the
third root of $F|_L$ must also be from $\mathbb{P}^2_k$, and thus from $C$.

\end{proof}


\end{document}
