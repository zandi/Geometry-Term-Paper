\documentclass[draft]{article}
\usepackage{amssymb,amsmath,amsthm,amsfonts}
\renewcommand{\abstractname}{Introduction} %change title of the abstract section
\newtheorem{mydef}{Definition}[section]
\newtheorem{lemma}{Lemma}[section]
\newtheorem{theorem}{Theorem}[section]

\begin{document}


\title{Algebraic Geometry}
\author{Zandi} %will use full information for assignment submission
\date{\today}
\maketitle

%TODO: remove me when finished
THIS IS A DRAFT. Expect inaccuracte and incomplete information.

\begin{abstract}
Algebraic Geometry is a field of mathematics with roots as far back as
at least the 19\textsuperscript{th} century, and with major advancement and applications
in recent years. At its core, Algebraic Geometry is concerned with 
studying the sets of zeroes of a system of polynomials, in particular 
their algebraic structure. Though real polynomials are surely the first
that come to mind, much of the power in Algebraic Geometry comes from the
generalization to polynomials with coefficients in arbitrary fields.\\

The undergraduate mathematics student would hopefully find this paper accessible,
but experience with algebra and geometry would help. A ``Basics" section is included,
but will not be exhaustive. As unhelpful as it seems, it's recommended to 
do independent research as necessary using Wikipedia, YouTube (videotaped lectures)
and the referenced textbooks.

This paper draws heavily from ``Undergraduate Algebraic Geometry" by Miles Reid, 
with the primary addition being annotated proofs. Where Reid's proofs
left details to the reader, these details are mostly provided, although
the proofs remain the same.
\end{abstract}

%\section{Basics}
Here's where I'll put all the theorems and definitions that
are simply necessary. I'll definitely talk about projective
space here, and homogenous equations.

\begin{mydef}
Given a Field $k$, the $\emph{Projective Space of Degree n over k}$, denoted $\mathbb{P}^n_k$
is the quotient $(k^{n+1} \setminus \{\bold{0}\}) / \sim$, where $\sim$ is the equivalence relation 
defined as 
\[
(x_0, ..., x_n) \sim (y_0, ..., y_n)~if~\exists \lambda \in k~s.t.~\lambda \neq 0,
(x_0, ..., x_n) = (\lambda y_0, ..., \lambda y_n)
\]
Given this, points in projective space are often denoted $P = [ x_0 : ... : x_n]$,
with the colons and brackets indicating this is an equivalence class. So then,
$[ y_0: ... : y_n ] = P = [ x_0 : ... : x_n ]$ if $(y_0, ..., y_n) \sim (x_0, ..., x_n)$.
\end{mydef}
The definition is abstract to purposely highlight the fact that, 
given a field $k$ we can create a projective space.
Effectively, the projective space is the set of all lines in 
$k^n$ through the origin. This is also why the projective space
has one fewer dimension than its corresponding vector space.

So, the ``points" in projective space are lines in the corresponding
vector space. Another interesting feature of projective space is
that parallel lines intersect at a ``point at infinity". This can
be thought of as making partitions of lines in the projective space
pased on parallelism, as if we were defining an equivalence relation.
Now, each equivalence class is assigned a unique ``point at infinity"
that each line in the equivalence class intersects. Thus, if two lines
are parallel, they intersect at some point at infinity. Likewise, if
two lines aren't parallel, they'll have their own points at infinity,
but they won't be the same, thus they will only have the usual intersection
normally expected of non-parallel lines.

This may seem very strange, but can be illustrated with a stereographic
projection. Suppose we're in $\mathbb{R}^3$, and picture two things:
the plane $z=1$ and a sphere of radius 1 centered at the origin.
Now, we can map any point $p$ on $z=1$ to a unique point on the upper
hemisphere of our sphere; simply note that the line passing through
$p$ and the origin $(0, 0, 0)$ intersects the specified hemisphere
precisely once at some point $q$. Knowing this, we can see that lines
on our plane $z=1$ are mapped to great circles on our sphere. In the
case of two parallel lines, they will never intersect in $z=1$, but
the great circles they're mapped to will intersect on the equator of 
the hemisphere.

\begin{mydef}
A function $f(x)$ is \emph{Homogenous of Degree n} if, for some 
constant $\lambda$, we have $f(\lambda x) = \lambda ^n f(x)$.
If $f$ is a homogenous polynomial of degree n, it is known as a \emph{form of degree n}.
\end{mydef}

Homogenous functions are particularly useful when considering their
sets of zeroes in projective space. If $f$ is homogenous of degree
n, and some point $x = (x_0, ..., x_n)$ is a zero of $f$, then
$f(\lambda x) = \lambda^n f(x) = \lambda^n 0 = 0$. Thus, we can say
that, if $f$ is homogenous, then we can indeed have zeroes of $f$ in $\mathbb{P}^n_k$,
assuming k is algebraically closed.

%this seemed like a good definition, but is overly broad. need to find 
%a more precise definition.
%\begin{mydef}
%For the purposes of this paper, a \emph{line} is a set whose points can be parameterised in terms
%of a single variable. That is, for some subset $l$ of our space, there exist functions $f_n$ s.t.
%\[
%	l = \{x : x = (x_1, x_2, \dots, x_n)\}
%\]
%where
%\[
%	(x_1, x_2, \dots, x_n) = (f_1(t), f_2(t), \dots, f_n(t))
%\]
%\end{mydef}
%
%As an illustrative example, in $\mathbb{R}^2$ we know that 
%$l = \{(x,mx+b):x\in\mathbb{R}\}$ is a line, and we can see that
%$l$ is fully parameterized in terms of $x$. Likewise, in $\mathbb{R}^n$,
%if we're given two points $a, b \in \mathbb{R}^n$ we can uniquely
%determine the line $l$ s.t. $a \in l, b \in l$ as follows:
%\[
%	l = \{ (1-t)a + tb : t \in \mathbb{R}\}
%\]
%This may cause some initial confusion when considering lines in projective
%spaces, such as $\mathbb{P}^2$. However, simply remember that the points in
%$\mathbb{P}^2$ are equivalence classes of some affine space of dimension $3$,
%such as $\mathbb{R}^3$ or $k^3$. Using a restriction such as $Z=1$ for
%$(X,Y,Z) \in \mathbb{P}^2$ can help with this.
%
%Of particular importance is that, when in $\mathbb{P}^2_k$, all lines will have
%some linear form whose set of roots determines the line.

\begin{mydef}
\[
S_d = \{\text{ forms of degree d in $(X, Y, Z)$}\}
\]
So, any element $F$ of $S_d$ can be written in a unique way as:
\[
F = \Sigma _{\substack{i+j+k'=d \\ i,j,k' > 0}}~a_{i,j,k'} X^i Y^j Z^{k'},~a_{i,j,k'} \in k
\]
\end{mydef}
So, $S_d$ has a basis of 
%improve formatting later...
\begin{align*}
\{&X^d, X^{d-1}Y, X^{d-2}Y^2, \dots\\
&X^{d-1}Z, X^{d-2}YZ, X^{d-3}Y^2Z, \dots\\
&X^{d-2}Z^2, X^{d-3}YZ^2, X^{d-4}Y^2Z^2, \dots\}
\end{align*}
Also, it can be seen that $\dim S_d = {d+2 \choose 2}$. On its own
$S_d$ isn't incredibly interesting, but we can make use of it when
reasoning about certain sets of geometric objects.

\begin{mydef}
for $p_1, \dots, p_n \in \mathbb{P}^2$, let
\[
	S_d(p_1, \dots, p_n) = \{ F \in S_d :~F(p_i) = 0,~i=1,\dots,n\}
\]
\end{mydef}

\section{Basics}
Here's where I'll put all the theorems and definitions that
are simply necessary. I'll definitely talk about projective
space here, and homogenous equations.

\begin{mydef}
Given a Field $k$, the $\emph{Projective Space of Degree n over k}$, denoted $\mathbb{P}^n_k$
is the quotient $(k^{n+1} \setminus \{\bold{0}\}) / \sim$, where $\sim$ is the equivalence relation 
defined as 
\[
(x_0, ..., x_n) \sim (y_0, ..., y_n)~if~\exists \lambda \in k~s.t.~\lambda \neq 0,
(x_0, ..., x_n) = (\lambda y_0, ..., \lambda y_n)
\]
Given this, points in projective space are often denoted $P = [ x_0 : ... : x_n]$,
with the colons and brackets indicating this is an equivalence class. So then,
$[ y_0: ... : y_n ] = P = [ x_0 : ... : x_n ]$ if $(y_0, ..., y_n) \sim (x_0, ..., x_n)$.
\end{mydef}
The definition is abstract to purposely highlight the fact that, 
given a field $k$ we can create a projective space.
Effectively, the projective space is the set of all lines in 
$k^n$ through the origin. This is also why the projective space
has one fewer dimension than its corresponding vector space.

So, the ``points" in projective space are lines in the corresponding
vector space. Another interesting feature of projective space is
that parallel lines intersect at a ``point at infinity". This can
be thought of as making partitions of lines in the projective space
pased on parallelism, as if we were defining an equivalence relation.
Now, each equivalence class is assigned a unique ``point at infinity"
that each line in the equivalence class intersects. Thus, if two lines
are parallel, they intersect at some point at infinity. Likewise, if
two lines aren't parallel, they'll have their own points at infinity,
but they won't be the same, thus they will only have the usual intersection
normally expected of non-parallel lines.

This may seem very strange, but can be illustrated with a stereographic
projection. Suppose we're in $\mathbb{R}^3$, and picture two things:
the plane $z=1$ and a sphere of radius 1 centered at the origin.
Now, we can map any point $p$ on $z=1$ to a unique point on the upper
hemisphere of our sphere; simply note that the line passing through
$p$ and the origin $(0, 0, 0)$ intersects the specified hemisphere
precisely once at some point $q$. Knowing this, we can see that lines
on our plane $z=1$ are mapped to great circles on our sphere. In the
case of two parallel lines, they will never intersect in $z=1$, but
the great circles they're mapped to will intersect on the equator of 
the hemisphere.

\begin{mydef}
A function $f(x)$ is \emph{Homogenous of Degree n} if, for some 
constant $\lambda$, we have $f(\lambda x) = \lambda ^n f(x)$.
If $f$ is a homogenous polynomial of degree n, it is known as a \emph{form of degree n}.
\end{mydef}

Homogenous functions are particularly useful when considering their
sets of zeroes in projective space. If $f$ is homogenous of degree
n, and some point $x = (x_0, ..., x_n)$ is a zero of $f$, then
$f(\lambda x) = \lambda^n f(x) = \lambda^n 0 = 0$. Thus, we can say
that, if $f$ is homogenous, then we can indeed have zeroes of $f$ in $\mathbb{P}^n_k$,
assuming k is algebraically closed.

%this seemed like a good definition, but is overly broad. need to find 
%a more precise definition.
%\begin{mydef}
%For the purposes of this paper, a \emph{line} is a set whose points can be parameterised in terms
%of a single variable. That is, for some subset $l$ of our space, there exist functions $f_n$ s.t.
%\[
%	l = \{x : x = (x_1, x_2, \dots, x_n)\}
%\]
%where
%\[
%	(x_1, x_2, \dots, x_n) = (f_1(t), f_2(t), \dots, f_n(t))
%\]
%\end{mydef}
%
%As an illustrative example, in $\mathbb{R}^2$ we know that 
%$l = \{(x,mx+b):x\in\mathbb{R}\}$ is a line, and we can see that
%$l$ is fully parameterized in terms of $x$. Likewise, in $\mathbb{R}^n$,
%if we're given two points $a, b \in \mathbb{R}^n$ we can uniquely
%determine the line $l$ s.t. $a \in l, b \in l$ as follows:
%\[
%	l = \{ (1-t)a + tb : t \in \mathbb{R}\}
%\]
%This may cause some initial confusion when considering lines in projective
%spaces, such as $\mathbb{P}^2$. However, simply remember that the points in
%$\mathbb{P}^2$ are equivalence classes of some affine space of dimension $3$,
%such as $\mathbb{R}^3$ or $k^3$. Using a restriction such as $Z=1$ for
%$(X,Y,Z) \in \mathbb{P}^2$ can help with this.
%
%Of particular importance is that, when in $\mathbb{P}^2_k$, all lines will have
%some linear form whose set of roots determines the line.

\begin{mydef}
\[
S_d = \{\text{ forms of degree d in $(X, Y, Z)$}\}
\]
So, any element $F$ of $S_d$ can be written in a unique way as:
\[
F = \Sigma _{\substack{i+j+k'=d \\ i,j,k' > 0}}~a_{i,j,k'} X^i Y^j Z^{k'},~a_{i,j,k'} \in k
\]
\end{mydef}
So, $S_d$ has a basis of 
%improve formatting later...
\begin{align*}
\{&X^d, X^{d-1}Y, X^{d-2}Y^2, \dots\\
&X^{d-1}Z, X^{d-2}YZ, X^{d-3}Y^2Z, \dots\\
&X^{d-2}Z^2, X^{d-3}YZ^2, X^{d-4}Y^2Z^2, \dots\}
\end{align*}
Also, it can be seen that $\dim S_d = {d+2 \choose 2}$. On its own
$S_d$ isn't incredibly interesting, but we can make use of it when
reasoning about certain sets of geometric objects.

\begin{mydef}
for $p_1, \dots, p_n \in \mathbb{P}^2$, let
\[
	S_d(p_1, \dots, p_n) = \{ F \in S_d :~F(p_i) = 0,~i=1,\dots,n\}
\]
\end{mydef}

\section{Conic Sections}

Use conic sections to illustrate some basics here
and provide necessary theorems/tools for elliptic curve
section. Will draw heavily from the book

\section{Bezout's Theorem}
Bezout's theorem will be important later, but we
probably won't be able to fully prove it here. However,
the book does prove the limited cases that we'll most rely on.

\begin{theorem}
\emph{Bezout's Theorem}:\\

If $C$ and $D$ are plane curves of degree $m$ and $n$ (respectively)
then $C \cap D$ contains precisely $mn$ points, provided:
\begin{align*}
&\text{The field we're working over is algebraically closed} \tag{i}\\
&\text{Points of intersection are counted with multiplicity} \tag{ii}\\
&\text{We work in $\mathbb{P}^2_k$ to account for points at infinity} \tag{iii}\\
\end{align*}
\end{theorem}

Instead of proving the general theorem, we'll only cover a special 
case of particular importance to us; intersections involving
lines and conics.

\begin{theorem}
Let $L \subset \mathbb{P}^2_k$ be a line (respectively, $C \subset \mathbb{P}^2_K$ a non-degenerate conic)
and $D \subset \mathbb{P}^2_k$ a curve defined by $D:(G_d(X,Y,Z)=0)$ where $G$ is
a form of degree $d$ in the 3 variables $X,Y,Z$. Also, assume that $L \not\subset D$ (resp. $C \not\subset D$).
Then
\[
|L \cap D| \le d
\]
and respectively
\[
|C \cap D| \le 2d
\]
And in fact, once accounting for multiplicity, points at infinity, and if $k$ is algebraically
closed, we have equality.
\end{theorem}



\section{Elliptic Curves}

One of the more interesting sections. Definitely need to
cover proving the given construction on the non-singuler elliptic
curve creates a group. Should also cover how in practice for crypto
the identity is chosen as the point at infinity, allowing for 
much cheaper computation.

Current strategy is to develop this section, then determine
dependencies for understanding, and stuff those in the
earlier sections.

\hrulefill

\begin{mydef}
\label{def:ellipticgroup}
\emph{Group Law on a Plane Cubic:}
Suppose $k \subset \mathbb{C}$ is a subfield of $\mathbb{C}$,
and $F \in k[X,Y,Z]$ is a cubic form defining a non-empty plane
curve $\mathfrak{C}:(F=0) \subset \mathbb{P}^2_k$. Additionally, assume that
$F$ satisfies the following:
\begin{align*}
F \text{ is irreducible, so $\mathfrak{C}$ won't contain a line or conic}&\tag{a}\\
\forall P \in \mathfrak{C},~\exists!~line~L \subset \mathbb{P}^2_k~s.t.~ P
\text{ is a repeated zero of } F|_L& \tag{b}\\
\end{align*}

Now, fix any point $O \in \mathfrak{C}$ and make the following construction:
\begin{align*}
&\text{for } A \in \mathfrak{C}, \text{ let } \bar{A} \text{ be the third point of intersection of $\mathfrak{C}$ with $\overleftrightarrow{OA}$} \tag{i}\\
&\text{for } A, B \in \mathfrak{C} \text{ let } R \text{ be the third point of intersection of $\mathfrak{C}$ with $\overleftrightarrow{AB}$}\\
&\text{ and define $A+B = \bar{R}$}\tag{ii}\\
\end{align*}
\end{mydef}

Geometrically, the uniqueness in (b) says that $\mathfrak{C}$ should be non-singular,
and we can see that this line $L$ is in fact the tangent line to $\mathfrak{C}$ at $P$.
Likewise, (a) prevents somewhat degenerate cases.

As you may have noticed, we've defined an operation on points of this
cubic curve $\mathfrak{C}$, so naturally we're curious what sort of structure then
arises. It turns out that this actually defines an Abelian Group on $\mathfrak{C}$,
as we will now prove.

\begin{theorem}
The above construction (Definition \ref{def:ellipticgroup}) defines
an Abelian Group $(\mathfrak{C},+)$ with $O$ as the identity element.
\end{theorem}

\begin{proof}
Well, to prove this is an abelian group, we have a few conditions we must satisfy:
\begin{enumerate}
\item Well-definedness of the operation and inverses
\item Existence of an identity element
\item Existence of unique inverses
\item Commutativity
\item Associativity
\end{enumerate}

\emph{Well-Definedness:}\\
Let $P, Q \in \mathfrak{C}$. Then, we have two simple cases:\\

case 1:  $P \ne Q$. Then $\exists! L \subset \mathbb{P}^2_k$ such that $L = \overleftrightarrow{PQ}$.\\

case 2: $P = Q$. Then, part (b) of the definition gives that $\exists! L \subset \mathbb{P}^2_k$ such that
$P$ is a repeated root of $F|_L$.\\

Note, however, that in both cases $F|_L$ is a cubic form in 2 variables.
So, as $F|_L$ is a 3rd degree form, it has precisely 3 roots, 2 of which
are the $k$-valued points $P$ and $Q$. Well, $P \in \mathbb{P}^2_k,~Q\in \mathbb{P}^2_k$,
so $F|_L$ is the product of 3 linear terms from $k[X,Y,Z]$. Thus, the
third root of $F|_L$ is certainly defined, and must also be from $\mathbb{P}^2_k$, and of course from $\mathfrak{C}$.

So, given any two points on $\mathfrak{C}$, the line they determine intersects $\mathfrak{C}$ at some third point.
Thus our construction, which relies on such a third point existing, is well defined. That is,
Given any $A \in \mathfrak{C},~\bar{A}$ is a unique element of $\mathfrak{C}$, and given any $A, B \in \mathfrak{C},~A+B$ 
is a unique point of $\mathfrak{C}$.\\

\emph{Commutativity:}\\
Well, need to show that $A+B = B+A$. Miles Reid chooses to ``leave this to the reader", 
and likewise we won't elaborate on this. Simply note that $\overleftrightarrow{AB} = \overleftrightarrow{BA}$,
so they share the same 3rd point of intersection with $\mathfrak{C}$.\\

\emph{Identity:}\\
Let $A \in \mathfrak{C}$ be some arbitrary point of $\mathfrak{C}$, and consider $O+A$.
Well, by our construction, we first find the third point
of intersection of $\overleftrightarrow{OA}$ with $\mathfrak{C}$, which by (i)
is $\bar{A}$. From here, to find $O+A$ we must now find the
third point of intersection of $\overleftrightarrow{O\bar{A}}$ with $\mathfrak{C}$. However, note
that $O,~A,~\bar{A}$ are all collinear. Thus, this third point
of intersection of $\overleftrightarrow{O\bar{A}}$ with $\mathfrak{C}$ is $A$. Hence, $O+A=A,~\forall A \in \mathfrak{C}$.\\

\emph{Inverses:}\\
Let $L$ be the line tanget to $\mathfrak{C}$ at $O$. That is, $L$ is the unique line s.t.
$F|_L$ has $O$ as a repeated zero. Now, let $\bar{O}$ be the 3rd root of $F|_L$.
Now, given any point $A \in \mathfrak{C}$, we can see that the third point of intersection
of $\overleftrightarrow{A\bar{O}}$ with $\mathfrak{C}$ is the inverse of $A$.

Call this third point of intersection $A'$. Now, consider $A+A'$. By definition,
$A,~A',~\bar{O}$ are all collinear, so the third point of intersection of 
$\overleftrightarrow{AA'}$ with $\mathfrak{C}$ is $\bar{O}$. Now, by how we defined
$\bar{O}$, we know that the third point of intersection of $\mathfrak{C}$ with $\overleftrightarrow{O\bar{O}}$
is $O$, as $O$ is a repeated root of this line. Thus, $A+A' = O$, so in the
future we'll refer to such an $A'$ as $A^{-1}$.

Finally, note that as $A^{-1}$ was picked as the third point of intersection of
$\mathfrak{C}$ with $\overleftrightarrow{A\bar{O}}$, it is well-defined.\\

\emph{Associativity:}\\
Associativity is by far the most difficult part of this proof. We'll first prove 
Associativity for one case, and then reduce the general case to this special case.\\

Suppose that $A,B,C \in \mathfrak{C}$. We want to show that $(A+B)+C = A+(B+C)$.
Well, in the construction of $(A+B)+C$ we use the 4 lines
\[
L_1: \overleftrightarrow{ABR},~L_2: \overleftrightarrow{RO\bar{R}},~L_3: \overleftrightarrow{C\bar{R}S},~L_4: \overleftrightarrow{SO\bar{S}}
\]
Likewise, the construction of $A+(B+C)$ involves
\[
M_1: \overleftrightarrow{BCQ},~M_2: \overleftrightarrow{QO\bar{Q}},~M_3: \overleftrightarrow{A\bar{Q}S'},~M_4: \overleftrightarrow{S'O\bar{S'}}
\]

Clearly, $(A+B)+C = A+(B+C) \iff \bar{S}=\bar{S'} \iff S=S'$. We'll prove $S=S'$ by inspecting the two cubics
\[
	D_1 = L_1 + M_2 + L_3 \text{ and } D_2 = M_1 + L_2 + M_3
\]
\end{proof}


\end{document}
