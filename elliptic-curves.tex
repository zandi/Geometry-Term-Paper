\section{Elliptic Curves}

One of the more interesting sections. Definitely need to
cover proving the given construction on the non-singuler elliptic
curve creates a group. Should also cover how in practice for crypto
the identity is chosen as the point at infinity, allowing for 
much cheaper computation.

Current strategy is to develop this section, then determine
dependencies for understanding, and stuff those in the
earlier sections.

\hrulefill

\begin{mydef}
\label{def:ellipticgroup}
\emph{Group Law on a Plane Cubic:}
Suppose $k \subset \mathbb{C}$ is a subfield of $\mathbb{C}$,
and $F \in k[X,Y,Z]$ is a cubic form defining a non-empty plane
curve $C:(F=0) \subset \mathbb{P}^2_k$. Additionally, assume that
$F$ satisfies the following:
\begin{align*}
F \text{ is irreducible, (C won't contain a line or conic)}&\tag{a}\\
\forall P \in C,~\exists! L \subset \mathbb{P}^2_k~s.t.~ P
\text{ is a repeated zero of } F|_L& \tag{b}\\
\end{align*}

Now, fix any point $O \in C$ and make the following construction:
\begin{align*}
&\text{for } A \in C, \text{ let } \bar{A} \text{ be the third point of intersection of $C$ with $\overleftrightarrow{OA}$} \tag{i}\\
&\text{for } A, B \in C \text{ let } R \text{ be the third point of intersection of $C$ with $\overleftrightarrow{AB}$}\\
&\text{ and define $A+B = \bar{R}$}\tag{ii}\\
\end{align*}
\end{mydef}

Geometrically, the uniqueness in (b) says that $C$ should be non-singular,
and we can see that this line $L$ is in fact the tangent line to $C$ at $P$.
Likewise, (a) prevents somewhat degenerate cases.

As you may have noticed, we've defined an operation on points of this
cubic curve $C$, so naturally we're curious what sort of structure then
arises. It turns out that this actually defines an Abelian Group on $C$,
as we will now prove.

\begin{theorem}
The above construction in Definition \ref{def:ellipticgroup} defines
an Abelian Group $(C,+)$ with $O$ as the identity element.
\end{theorem}

\begin{proof}
Well, to prove this is an abelian group, we have a few conditions we must satisfy:
\begin{enumerate}
\item Well-definedness of the operation and inverses
\item Existence of an identity element
\item Existence of unique inverses
\item Commutativity
\item Associativity
\end{enumerate}

\emph{Well-Definedness:}\\
Let $P, Q \in C$. Then, we have two simple cases:\\

case 1:  $P \ne Q$. Then $\exists! L \subset \mathbb{P}^2_k$ such that $L = \overleftrightarrow{PQ}$.\\

case 2: $P = Q$. Then, part (b) of the definition gives that $\exists! L \subset \mathbb{P}^2_k$ such that
$P$ is a repeated root of $F|_L$.

Note, however, that in both cases $F|_L$ is a cubic form in 2 variables. $L$ is 
a line, and thus defined as the set of zeroes of some linear form $l \in k[X,Y,Z]$. As is common, we've
held $Z$ constant and non-zero to simplify our work, and use the common substitution
$x=X/Z,~y=Y/Z$. Thus, as $F|_L = F \circ l$, $F|_L$ can only be of 3rd degree,
and is still homogenous. ($F(l(\lambda x)) = F(\lambda l(x)) = \lambda^3 F(l(x))$).

So, as $F|_L$ is a 3rd degree form, it has precisely 3 roots, 2 of which
are known to be $P$ and $Q$. Well, $P \in \mathbb{P}^2_k,~Q\in \mathbb{P}^2_k$,
so $F|_L$ is the product of 3 linear terms from $k[X,Y,Z]$. Thus, the
third root of $F|_L$ must also be from $\mathbb{P}^2_k$, and thus from $C$.

So, given any two points on $C$, the line they determine intersects $C$ at some unique third point.
Thus our construction, which relies on such a third point existing, is well defined. That is,
Given any $A \in C,~\bar{A}$ is a unique element of $C$, and given any $A, B \in C,~A+B$ 
is a unique point of $C$.\\

\emph{Identity:}\\
Let $A \in C$ be some arbitrary point of $C$, and consider $O+A$.
Well, by our construction, we first find the third point
of intersection of $\overleftrightarrow{OA}$ with $C$, which by (i)
is $\bar{A}$. From here, to find $O+A$ we must now find the
third point of intersection of $\overleftrightarrow{O\bar{A}}$ with $C$. However, note
that $O,~A,~\bar{A}$ are all collinear. Thus, this third point
of intersection of $\overleftrightarrow{O\bar{A}}$ with $C$ is $A$. Hence, $O+A=A,~\forall A \in C$.

\end{proof}
