\section{Conic Sections}

Conic sections provide perhaps the simplest geometric
figures to study in algebraic geometry. They are fairly
well understood within $\mathbb{R}^2$, and extend quite beautifully
into $\mathbb{P}^2$. One example of this is examining 
the classification of conics in both $\mathbb{R}^2$ and
$\mathbb{P}^2$. While both have their degenerate cases,
it's interesting to note that in $\mathbb{R}^2$ there
are 3 non-degenerate cases of a conic section;
the ellipse, hyperbola, and parabola. However, in
$\mathbb{P}^2$ these can all be described in a single
non-degenerate case, thanks to the point at infinity of
projective space. Also, algebraic geometry can inform
the historically interesting execercise of finding
rational or integer roots for the pythagorean equation
$X^2 + Y^2 = Z^2$.

However, we will only briefly cover conic sections, and even
then primarily for their application to forthcoming sections.
Let's begin by defining what a conic actually is.

\begin{mydef}
A \emph{Conic} in $\mathbb{R}^2$ is a plane curve given by the quadratic
equation
\[
q(x,y) = ax^2 + bxy + cy^2 + dx + ey + f = 0
\]
\end{mydef}

To ``adapt" this definition into projective space $\mathbb{P}^2$, we simply
note that the inhomogenous polynomial $q$ has an associated
homogenous polynomial 
\[
Q(X,Y,Z) = aX^2 + bXY + cY^2 + dXZ + eYZ + fZ^2
\]
Miles Reid recommends thinking about this as a bijection $q \leftrightarrow Q$ by
\[
q(x,y) = Q(\frac{X}{Z},\frac{Y}{Z},\frac{Z}{Z}),~Z\ne 0
\]
Thus, a conic in $\mathbb{P}^2$ would be determined by the zeroes of $Q$.

\begin{theorem}
Let $k$ be any field of characteristic $\ne 2$. Then, any quadratic form $Q$
in $(X,Y,Z)$ has an associated $3 \times 3$ symmetric matrix $A$ s.t.
\[
x^TAx =  Q(X,Y,Z),~x=\begin{bmatrix}
x\\
y\\
z
\end{bmatrix}
\]
explicitly, if 
\[
Q(X,Y,Z) = aX^2 + 2bXY + cY^2 + 2dXZ + 2eYZ + fZ^2
\]
Then
\[
A = \begin{bmatrix}
a & b & d\\
b & c & e\\
d & e & f
\end{bmatrix}
\]
and the quadratic form (and hence the conic it defines) is non-degenerate
if the matrix $A$ is non-degenerate (non-singular).
\end{theorem}

Note that the matrix $A$ from the above theorem is symmetric. Thus, we can
apply the Spectral Theorem to see that we can diagonalize it using some
orthogonal matrix. This effectively allows us to transform any non-degenerate 
quadratic form into a homogenous quadratic form through a change of basis.

\begin{theorem}
Let $Q$ be a quadratic form of $X, Y, Z$. Then, there exists a change of basis $(X,Y,Z) \mapsto (X',Y',Z')$ s.t.
$Q(X,Y,Z) = Q'(X',Y',Z')$ where
\[
Q'(X',Y',Z') = \lambda_1 X'^2 + \lambda_2 Y'^2 + \lambda_3 Z'^2
\]
As a corollary, all non-degenerate conics in $\mathbb{P}^2$ are projectively
equivalent to the conic defined by $XZ = Y^2$. That is, there is some
projective transformation which takes an arbitrary conic $C$ to the conic 
defined by $XZ = Y^2$.
\end{theorem}
\begin{proof}
$Q$ is a quadratic form,
thus $Q$ has an associated symmetric matrix 
$A$ s.t. $Q(x) = x^TAx$. As $A$ is symmetric, the Spectral Theorem gives that
\[
\exists P:~P\text{ orthogonal},~P^TAP\text{ is diagonal}
\]
and in fact, the diagonal entries of $B=P^TAP$ are the eigenvalues of $A$. Thus,
it is easy to show, using the transformation $x=Px'$, that
\[
Q(X,Y,Z) = x^TAx = (Px')^TAPx' = x'^TP^TAPx' = x'^TBx'
\]
Now, letting the eigenvalues of $A$ be $\lambda_1, \lambda_2, \lambda_3$ we see that
\[
x'^TBx' = \lambda_1 X'^2 + \lambda_2 Y'^2 + \lambda_3 Z'^2 = Q'(X',Y',Z')
\]
Thus, $Q(X,Y,Z) = Q'(X',Y',Z')$.
\end{proof}
Thus, conic sections in $\mathbb{P}^2$ have an interesting sort of 
unifying characteristic when regarded under projective transformations.
Not only is there one case of a non-degenerate conic instead of 3,
but any non-degenerate conic is ultimately just a projective transformation
of $XZ=Y^2$!. This will come in handy in just a few pages.
