\section{Conclusion}

In conclusion, algebraic geometry is a very interesting
field with wide-reaching implications. Though my personal interest
in cryptography led me to focus on the group on
the elliptic curve, Bezout's Theorem perhaps has much wider
theoretic application, and the impact of projective space
on classifying conic sections is simply remarkable.

If you found this paper interesting, it's highly
recommended you somehow obtain a copy of 
``Undergraduate Algebraic Geometry" by Miles Reid. This 
paper should serve as a useful synopsis of the first portion
of the book, particularly for some of the proofs which
are somewhat short on specifics. It's also recommended
to brush up on commutative algebra for more advanced material.
Specifically,
``Commutative Algebra: with a View Toward Algebraic Geometry"
by David Eisenbud (published by Springer) would be useful.

Finally, something I was hoping to cover but couldn't is
regular polyhedron. Due to their highly symmetric nature,
regular polyhedron are naturally defined as the polyhedron
which are invariant in some fashion under certain transformations.
For example, Wikipedia defines this as ``A regular polyhedron is a polyhedron
whose symmetry group acts transitively on its flags", but without
having studied this subject myself, I'll refrain from
getting into details I don't yet understand. Worth noting,
though, is the strange relationship between the number of
regular polyhedron that exist, and the dimension we're considering.
Starting from 1, that sequence is $1,\infty,5,6,3,3,\dots$, and 
so on into arbitrarily large but finite dimension.

